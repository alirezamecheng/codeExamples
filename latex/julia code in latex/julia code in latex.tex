\documentclass{article}

\usepackage[T1]{fontenc}
\usepackage{beramono}
\usepackage{listings}
\usepackage[usenames,dvipsnames]{xcolor}

%%
%% Julia definition (c) 2014 Jubobs
%%

\lstdefinelanguage{Julia}%
  {morekeywords={abstract,break,case,catch,const,continue,do,else,elseif,%
      end,export,false,for,function,immutable,import,importall,if,in,%
      macro,module,otherwise,quote,return,switch,true,try,type,typealias,%
      using,while},%
   sensitive=true,%
%   alsoother={$},%
   morecomment=[l]\#,%
   morecomment=[n]{\#=}{=\#},%
   morestring=[s]{"}{"},%
   morestring=[m]{'}{'},%
}[keywords,comments,strings]%

\lstset{%
    language         = Julia,
    basicstyle       =\footnotesize\ttfamily,
    keywordstyle     = \bfseries\color{blue},
    stringstyle      = \color{magenta},
    commentstyle     = \color{ForestGreen},
    showstringspaces = false,
    %added later
    breaklines=true, % Automatic line breaking (prevents code from protruding outside the box)
    frame=single, % Single frame around code
    numbers=left, % Line numbers on left
    firstnumber=1, % Line numbers start with line 1
    numberstyle=\tiny\color{Blue}, % Line numbers are blue and small
    stepnumber=1 % Line numbers go in steps of 5
    %end added later
}

\begin{document}
\begin{lstlisting}
#= This is a code sample for the Julia language
(adapted from http://julialang.org) =#
function mandel(z)
    c = z
    maxiter = 80
    for n = 1:maxiter
        if abs(z) > 2
            return n-1
        end
        z = z^2 + c
    end
    return maxiter
end

function helloworld()
    println("Hello, World!") # Bye bye, MATLAB!
end

function randmatstat(t)
    n = 5
    v = zeros(t)
    w = zeros(t)
    for i = 1:t
        a = randn(n,n)
        b = randn(n,n)
        c = randn(n,n)
        d = randn(n,n)
        P = [a b c d]
        Q = [a b; c d]
        v[i] = trace((P.'*P)^4)
        w[i] = trace((Q.'*Q)^4)
    end
    std(v)/mean(v), std(w)/mean(w)
end
\end{lstlisting}
\end{document}